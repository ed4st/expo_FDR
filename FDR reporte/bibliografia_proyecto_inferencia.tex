@article{Miller_2001,
   title={Controlling the False-Discovery Rate in Astrophysical Data Analysis},
   volume={122},
   ISSN={0004-6256},
   url={http://dx.doi.org/10.1086/324109},
   DOI={10.1086/324109},
   number={6},
   journal={The Astronomical Journal},
   publisher={American Astronomical Society},
   author={Miller, Christopher J. and Genovese, Christopher and Nichol, Robert C. and Wasserman, Larry and Connolly, Andrew and Reichart, Daniel and Hopkins, Andrew and Schneider, Jeff and Moore, Andrew},
   year={2001},
   month={Dec},
   pages={3492–3505}
}

@article{Benjamini1995,
  abstract = {The common approach to the multiplicity problem calls for controlling
	the familywise error rate (FWER). This approach, though, has faults,
	and we point out a few. A different approach to problems of multiple
	significance testing is presented. It calls for controlling the expected
	proportion of falsely rejected hypotheses-the false discovery rate.
	This error rate is equivalent to the FWER when all hypotheses are
	true but is smaller otherwise. Therefore, in problems where the control
	of the false discovery rate rather than that of the FWER is desired,
	there is potential for a gain in power. A simple sequential Bonferroni-type
	procedure is proved to control the false discovery rate for independent
	test statistics, and a simulation study shows that the gain in power
	is substantial. The use of the new procedure and the appropriateness
	of the criterion are illustrated with examples.},
  added-at = {2009-08-05T04:34:59.000+0200},
  author = {Benjamini, Yoav and Hochberg, Yosef},
  biburl = {https://www.bibsonomy.org/bibtex/27d9c780b64c0e5ff5bbf2c69ec14b11f/ebo},
  citeulike-article-id = {1042553},
  doi = {http://dx.doi.org/10.2307/2346101},
  interhash = {b38b0e6655978ad8c7d8455b175c2cbf},
  intrahash = {7d9c780b64c0e5ff5bbf2c69ec14b11f},
  journal = {Journal of the Royal Statistical Society Series B (Methodological)},
  keywords = {statistics},
  number = 1,
  pages = {289-300},
  posted-at = {2008-02-29 10:56:34},
  timestamp = {2009-08-05T04:34:59.000+0200},
  title = {Controlling the False Discovery Rate: A Practical and Powerful Approach
	to Multiple Testing},
  url = {http://dx.doi.org/10.2307/2346101},
  volume = 57,
  year = 1995
}


@article{cerebro,
author = {Marcus, Daniel S. and Wang, Tracy H. and Parker, Jamie and Csernansky, John G. and Morris, John C. and Buckner, Randy L.},
title = {Open Access Series of Imaging Studies (OASIS): Cross-Sectional MRI Data in Young, Middle Aged, Nondemented, and Demented Older Adults},
year = {2007},
issue_date = {September 2007},
publisher = {MIT Press},
address = {Cambridge, MA, USA},
volume = {19},
number = {9},
issn = {0898-929X},
url = {https://doi.org/10.1162/jocn.2007.19.9.1498},
doi = {10.1162/jocn.2007.19.9.1498},
abstract = {The Open Access Series of Imaging Studies is a series of magnetic resonance imaging data sets that is publicly available for study and analysis. The initial data set consists of a cross-sectional collection of 416 subjects aged 18 to 96 years. One hundred of the included subjects older than 60 years have been clinically diagnosed with very mild to moderate Alzheimer's disease. The subjects are all right-handed and include both men and women. For each subject, three or four individual T1-weighted magnetic resonance imaging scans obtained in single imaging sessions are included. Multiple within-session acquisitions provide extremely high contrast-to-noise ratio, making the data amenable to a wide range of analytic approaches including automated computational analysis. Additionally, a reliability data set is included containing 20 subjects without dementia imaged on a subsequent visit within 90 days of their initial session. Automated calculation of whole-brain volume and estimated total intracranial volume are presented to demonstrate use of the data for measuring differences associated with normal aging and Alzheimer's disease.},
journal = {J. Cognitive Neuroscience},
month = sep,
pages = {1498–1507},
numpages = {10}
}

@article{git_cerebro_2,
title = "Influence of multiple hypothesis testing on reproducibility in neuroimaging research: A simulation study and Python-based software",
journal = "Journal of Neuroscience Methods",
volume = "337",
pages = "108654",
year = "2020",
issn = "0165-0270",
doi = "https://doi.org/10.1016/j.jneumeth.2020.108654",
url = "http://www.sciencedirect.com/science/article/pii/S0165027020300765",
author = "Tuomas Puoliväli and Satu Palva and J. Matias Palva",
keywords = "False discovery rate, Family-wise error rate, Multiple hypothesis testing, Neurophysiological data, Python, Reproducibility",
abstract = "Background
Reproducibility of research findings has been recently questioned in many fields of science, including psychology and neurosciences. One factor influencing reproducibility is the simultaneous testing of multiple hypotheses, which entails false positive findings unless the analyzed p-values are carefully corrected. While this multiple testing problem is well known and studied, it continues to be both a theoretical and practical problem.
New method
Here we assess reproducibility in simulated experiments in the context of multiple testing. We consider methods that control either the family-wise error rate (FWER) or false discovery rate (FDR), including techniques based on random field theory (RFT), cluster-mass based permutation testing, and adaptive FDR. Several classical methods are also considered. The performance of these methods is investigated under two different models.
Results
We found that permutation testing is the most powerful method among the considered approaches to multiple testing, and that grouping hypotheses based on prior knowledge can improve power. We also found that emphasizing primary and follow-up studies equally produced most reproducible outcomes.
Comparison with existing method(s)
We have extended the use of two-group and separate-classes models for analyzing reproducibility and provide a new open-source software “MultiPy” for multiple hypothesis testing.
Conclusions
Our simulations suggest that performing strict corrections for multiple testing is not sufficient to improve reproducibility of neuroimaging experiments. The methods are freely available as a Python toolkit “MultiPy” and we aim this study to help in improving statistical data analysis practices and to assist in conducting power and reproducibility analyses for new experiments."
}

@PHDTHESIS{tesis_cimat_selecc,
  author = {Edgar Eduardo Rodríguez Mendoza},
  title =        "{Selección y Ordenamiento con
Aplicaciones en Genética}",
  school =       "Maestría en Ciencias con Especialidad en Probabilidad y Estadística. CENTRO DE INVESTIGACIÓN EN MATEMATICAS A.C.",
  year =         "2015",
  month =        "10",
}

