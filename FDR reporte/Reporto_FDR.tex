\documentclass[11pt,letterpaper]{article}
\usepackage[utf8]{inputenc}
\usepackage[T1]{fontenc}
\usepackage[spanish]{babel}
\usepackage{amsmath}
\usepackage{amsfonts}
\usepackage{amssymb}
\usepackage{graphicx}
\usepackage{lmodern}
\usepackage{xspace}
\usepackage{multicol}
\usepackage{hyperref}
\usepackage{float}
\usepackage{hyperref}
\usepackage{color}
\usepackage{framed}

\usepackage[left=2cm,right=2cm,top=2cm,bottom=2cm]{geometry}

\newcommand{\X}{\mathbb{X}}
\newcommand{\x}{\mathbf{x}}
\newcommand{\Y}{\mathbf{Y}}
\newcommand{\y}{\mathbf{y}}
\newcommand{\xbarn}{\bar{x}_n}
\newcommand{\ybarn}{\bar{y}_n}
\newcommand{\paren}[1]{\left( #1 \right)}
\newcommand{\llaves}[1]{\left\lbrace #1 \right\rbrace}
\newcommand{\barra}{\,\vert\,}
\newcommand{\mP}{\mathbb{P}}
\newcommand{\mE}{\mathbb{E}}
\newcommand{\mR}{\mathbb{R}}
\newcommand{\mJ}{\mathbf{J}}
\newcommand{\mX}{\mathbf{X}}
\newcommand{\mS}{\mathbf{S}}
\newcommand{\mA}{\mathbf{A}}
\newcommand{\unos}{\boldsymbol{1}}
\newcommand{\xbarnv}{\bar{\mathbf{x}}_n}
\newcommand{\abs}[1]{\left\vert #1 \right\vert}
\newcommand{\muv}{\boldsymbol{\mu}}
\newcommand{\mcov}{\boldsymbol{\Sigma}}
\newcommand{\vbet}{\boldsymbol{\beta}}
\newcommand{\veps}{\boldsymbol{\epsilon}}
\newcommand{\mcC}{\mathcal{C}}
\newcommand{\mcR}{\mathcal{R}}
\newcommand{\mcN}{\mathcal{N}}

\newcommand{\ceros}{\boldsymbol{0}}
\newcommand{\mH}{\mathbf{H}}
\newcommand{\ve}{\mathbf{e}}
\newcommand{\avec}{\mathbf{a}}
\newcommand{\res}{\textbf{RESPUESTA}\\}

\newcommand{\defi}[3]{\textbf{Definición:#3}}
\newcommand{\fin}{$\blacksquare.$}
\newcommand{\finf}{\blacksquare.}
\newcommand{\tr}{\text{tr}}
\newcommand*{\temp}{\multicolumn{1}{r|}{}}

\newcommand{\grstep}[2][\relax]{%
   \ensuremath{\mathrel{
       {\mathop{\longrightarrow}\limits^{#2\mathstrut}_{
                                     \begin{subarray}{l} #1 \end{subarray}}}}}}
\newcommand{\swap}{\leftrightarrow}

\newcommand{\gen}{\text{gen}}
\newtheorem{thmt}{Teorema:}
\newtheorem{thmd}{Definición:}
\newtheorem{thml}{Lema:}
\newtheorem{thme}{Ejemplo:}


\begin{document}
\begin{table}[ht]
\centering
\begin{tabular}{c}
\textbf{Comparaciones múltiples desde la perspectiva de ciencia de datos y}\\
\textbf{su relación con el False Discovery Rate.}\\
\textit{Reporte FDR: Enrique Santibáñez Cortés}
\end{tabular}
\end{table}
\section*{False Discovery Rate}
La tasa de falsos descubrimiento (FDR), fue propuesto por primera en . La propuesta de Benjamini and Hochberg [1995] se colocó como una de las piezas clave en la investigación relacionada con FDR. Lo anterior debido a que, hasta antes de dicha publicación, la mayor parte de la inferencia relacionada con PHM se hacía fundamentalmente con base en métodos relacionados con el control de FWER, o bien técnicas similares derivadas de modificaciones a la misma. Consiste de la proporción esperada de errores entre la hipótesis rechazadas. Formalmente, si definimos la variable aleatoria $Q$ como:
\begin{align} \label{e_fdr}
Q=\left\{ \begin{array}{cc}
V/R & R>0\\
0 & R=0
\end{array} \right.
\end{align}
Entonces, se tiene que 
\begin{align*}
FDR = \mE(Q)=\mE\left( \frac{V}{V+S}\right)=\mE\left( \frac{V}{R} \right).
\end{align*}
Pero como esta definido $Q$ tenemos que 
\begin{align}\label{igudaldad_Q}
FDR=\mE\left( \frac{V}{R} \right) &= \mE\left( \frac{V}{R}\left| R>0\right. \right)\mP(R>0)+\mE\left( \frac{V}{R}\left| R=0\right. \right)\mP(R=0)\\
&= \mE\left( \frac{V}{R}\left| R>0\right. \right)\mP(R>0).
\end{align}

Antes de la públicación de Benjamini, la mayor parte de la inferencia relacionada con PHM se hacía fundamentalmente con base en métodos relacionados con el control de FWER, o bien técnicas similares derivadas de modificaciones a la misma. FWER posee desventajas importantes que con el surgimiento de conjuntos de datos de mayor tamaño, por ejemplo en el contexto de genética, se hicieron más evidentes.\\
Esta tasa cambió el panorama de los test de hipótesis multiples, ya que incentivó el desarrollo de numerosas nuevas investigaciones centradas tanto en la búsqueda de nuevas tasas de error derivadas de (\ref{e_fdr}), así como de procedimientos alternativos, al propuesto por BH, para controlar la FDR. Algunas propiedades de esta tasa 
\begin{enumerate}
\item Si todas las hipótesis nulas son verdaderas, entonces controlar la FDR es equivalente a controlar la FWER. \\
\textbf{Demostración.} Si todas las hipótesis nulas son verdaderas entonces $V=R$. Si $V=0$ entonces $\frac{V}{R}=0$ y si $V>0$ entonces $\frac{V}{R}=0$, por lo que (utilizando el resultado de (\ref{igudaldad_Q}))
\begin{align*}
FDR &= \mE\left( \frac{V}{R}\left| V=0\right. \right)\mP(V=0)+\mE\left( \frac{V}{R}\left| V>0\right. \right)\mP(V>0)\\
&=0\times \mP(V=0)+1\times \mP(V\geq 1)\\
&= \mP(V\geq 1)\\
&= FWER.\ \ \finf
\end{align*}

\item  Por lo tanto, si controlamos el FDR (es decir, lo mantenemos por debajo de algún valor), entonces estamos controlando el FWER en el sentido débil. \\
\textbf{Demostración.} Si no todas las hipótesis nulas son verdaderas, entonces $V<R$ y  $\frac{V}{R}<1$, y esto implica que $\mE\left( \frac{V}{R} |V\geq 1\right)<1$. Ocupando lo anterior tenemos
\begin{align*}
FDR&= \mE\left( \frac{V}{R}\left| V=0\right. \right)\mP(V=0)+\mE\left( \frac{V}{R}\left| V\geq 1\right. \right)\mP(V\geq 1)\\
&=0\times \mP(V=0)+\mE\left( \frac{V}{R}\left| V\geq 1\right. \right)\mP(V\geq 1)\\
&<FWER.\ \ \finf
\end{align*}

\end{enumerate}
Para mostr punto 1como  El punto 2 es el más interesante, pues significa que cualquier procedimiento que controle la FWER también controla la FDR, pero no necesariamente al revés. 


\subsection{Procedimiento de control de la FDR de Benjamini y Hochberg(B-H))}
Considere la pruebas $H_1, H_2, ..., H_m$, basado en los p$-$values correspondientes $P_{(1)}, P_{(2)},..., P_{(m)}$. Sean $P_{(1)}\leq P_{(2)}\leq \cdots \leq  P_{(m)}$ los p$-$values están ordenados y denotar por la hipótesis nula $H_{(i)}$ correspondiente a $P_{(i)}$. Defina el siguiente procedimiento de prueba múltiple de Bonferroni$-$type:\\
\begin{align}
 \text{sea k la $i$ más grande para la cual }P_{(i)}\leq \frac{i}{m}q*;\\
 \text{luego rechaza todo }H_{(i)} = 1, 2, ..., k.
\end{align} 

\begin{framed}
    \begin{thmt} \label{t_1}
Para estadísticas de prueba independientes y para cualquier configuración de hipótesis nulas falsas, el procedimiento anterior controla el FDR en $q*$.
    \end{thmt}
\end{framed}
Prueba. El teorema se deriva del siguiente lema, cuya demostración se da en el apéndice A.

\begin{framed}
    \begin{thml} \label{l_1}
Para cualquier $0 <m_0 <m$ p-values independientes correspondientes a hipótesis nulas verdaderas, y para cualquier valor que puedan tomar los p-values de $m_1= m - m_0$ correspondientes a las hipótesis nulas falsas, el procedimiento de prueba múltiple definido por el procedimiento$(1)$ anterior satisface la desigualdad
\begin{equation}\label{2}
\mE (Q | P_{m_0+1}=p_1, ..., P_m = p_{m_1})\leq \frac{m_0}{m} q *
\end{equation}
Ahora, suponga que $m_1=m-m_0$ algunas de las hipótesis son falsas. Cualquiera que sea la distribución conjunta de $P_1'' ,\cdots, P_{m_1}''$, que corresponde a estas falsas hipótesis, integrando la desigualdad (\ref{2}) anterior obtenemos
\begin{equation*}
\mE(Q) <\frac{m_0}{m}q* <q *,
\end{equation*}
y el FDR está controlado.
    \end{thml}
\end{framed}
\textbf{Demostración del lema \ref{2}}. La prueba del lema es por inducción en $m$. Para el caso $m=1$ es inmediato, procedemos asumiendo que el lema es verdadero para cualquier $m'\leq m$, y demostremos que es válido para $m+1$. \\
Si $m_0$ es 0, todas las hipótesis nulas son falsas, por lo que $Q=0$ y 
\begin{align*}
\mE(Q|P_1=p_1,\cdots, P_m=p_m)=0 \leq \frac{m_0}{m+1}q*.
\end{align*} 
Si $m_0>0$, denotemos por $P_i, \ \ i=1,2,\cdots, m_0$ los valores $p$ correspondientes a las verdaderas hipótesis nulas y el mayor de estas $P_{(m_0)}'$. Estos son v.a. independientes $U(0,1)$. Para facilitar la notación asumamos que los $m_1$ p$-$values las hipótesis nulas falsas ordenadas, es decir, $p_1\leq p_2 \leq \cdots \leq p_{m_1}$ y denotemos a $j_0$ más grande de $m_1$ es decir $0\leq j\leq m_1$ que satisface
\begin{align*}\label{desigualdad}
p_j\leq \frac{m_0+j}{m+1}q*,
\end{align*}
y denotemos lo que esta a la derecha de la desigualdad (\ref{desigualdad}) por $p''$, es decir, $$p''=\frac{m_0+j}{m+1}q*.$$
Ahora, condicionando en (\ref{desigualdad}) $P_{(m0)}'=p$ tenemos que 
\begin{align}
\mE (Q | P_{m_0+1}=p_1, ..., P_m = p_{m_1})&=\int_0^1 \mE (Q | P_{(m0)}'=p, P_{m_0+1}=p_1, ..., P_m = p_{m_1}) f_{P_{(m_0)}}'(p)dp\\
&=\int_0^{p''} \mE (Q | P_{(m0)}'=p, P_{m_0+1}=p_1, ..., P_m = p_{m_1}) f_{P_{(m_0)}}'(p)dp+\\
&\int_{p''}^1 \mE (Q | P_{(m0)}'=p, P_{m_0+1}=p_1, ..., P_m = p_{m_1}) f_{P_{(m_0)}}'(p)dp
\end{align} 
donde $f_{P_{m_0}}=m_0p^{(m_0-1)}.$
Para la primera parte cuando $p\leq p''$. Como todas las hipótesis $m_0+j_0$ nulas son rechazadas, y $Q=m_0/(m_0+j_0)$. Entonces evaluando la primera integral, y usando (\ref{desigualdad})
\begin{align}\label{desigualdad_6}
\frac{m_0}{m_0+j_0}(p'')^{m_0} \leq \frac{m_0}{m_0+j_0} \frac{m_0+j_0}{m+1}q*(p'')^{m_0-1}=\frac{m_0}{m+1}q*(p'')^{m_0-1}.
\end{align}

Ahora, en la segunda parte de la integral (--), consideremos separar cada $p_{j_0}< p_j\leq P_{(m_0)}'=p<p_{j+1}$ y $p_{j_0}< p'' \leq P_{(m_0)}'=p<p_{j_0+1}$.  Es importante señalar que, debido a la forma en que se definen $j_0$ y $p"$, no se puede rechazar ninguna hipótesis como resultado de los valores de $p, p_{j + 1}, p_{j + 2},\cdots, p_{m_1}$. Por lo tanto, cuando todas las hipótesis son verdaderas y falsas se consideran juntas, y sus valores $p$ así ordenados, una hipótesis $H_{(i)}$ puede rechazarse solo si existe $k, i <k<m_0 + j-1$, para lo cual $p_{(k)} <\{k / ( m + 1)\} q *$, o equivalentemente 
\begin{align}\label{q_p_k}
\frac{p_{(k)}}{p}\leq \frac{k}{m_0+j-1} \frac{m_0+j-1}{(m+1)p}q^*.
\end{align}
Cuando se condiciona con $P_{(m_0)}'=p$, los $P_j'/p$ para $i=1,2, \cdots, m_0-1$ son $m_0-1$ v.a's independientes con distribución $U(0,1)$. y los $p_i/p$ para $i=1,2, \cdots,j$ corresponden al número de hipótesis nulas falsas entre 0 y 1. 
Usando la desigualdad (\ref{q_p_k}) para las $m_0+j-1=m'\leq $ hipótesis es equivalente usando que $P_{(i)}\leq \frac{i}{m}q*$, con cota $\{(m_0+j-1)/(m+1)p \}q*$. Aplicando ahora la hipótesis de inducción tenemos que 
\begin{align}\label{final}
\mE (Q | P_{(m0)}'=p, P_{m_0+1}=p_1, ..., P_m = p_{m_1}) \leq \frac{m_0-1}{m_0+j-1} \frac{m_0+j-1}{(m+1)p}q^* = \frac{m_0-1}{(m+1)p}q^*.
\end{align}
Considerando la desigualdad (\ref{final}) depende de $p$, pero no del segmento $p_j<p<p_{j+1}$ ppr lo cuál podemos integral y ocupando (\ref{desigualdad_6}) podemos concluir que 
\begin{align*}
\int_{p''}^1\mE (Q | P_{(m0)}'&=p, P_{m_0+1}=p_1, ..., P_m = p_{m_1})f_{P_{m_0}}'(p) \leq  \frac{m_0-1}{(m+1)p}q^*m_0p^{(m_0-1)}dp\\
&=\frac{m_0}{m+1}q*\int_{p''}^1(m_0-1)p^{(m_0-2)}dp=\frac{m_0}{m+1}q^*\{1-p''^{(m_0-1)} \}. \ \ \finf
\end{align*}

Cabe aclarar que una de las principales desventajas potenciales del algoritmo B-H tal y como se propuso en Benjamini and Hochberg [1995] yace en el supuesto de independencia entre las m hipótesis requerido para asegurar el control de la FDR. En respuesta a esa situación Benjamini and Yekutieli [2001] demuestran que el procedimiento B-H también puede ofrecer control de la FDR para los casos en los cuales las hipótesis están positivamente correlacionadas, como es el caso en una gran variedad de aplicaciones como en Génetica y Ecología. 



\subsection{Ejemplos de control}

\subsubsection*{Comparación de la FDR y FWER}
Se ha demostrado que la trombólisis con activador de plasminógeno de tipo tisular recombinante ($rt-PA)$ y activador de estreptoquinasa de plasminógeno anisoilado (APSAC) en el infarto de miocardio reduce la mortalidad. Neuhaus y col. (1992) investigaron los efectos de una nueva administración de carga frontal de rt$-$PA versus los obtenidos con un régimen estándar de APSAC. El estudio se pueden identificar cuatro familias de hipótesis, pero la que puede ser deseable el control de FDR ya que no se quiere concluir que el tratamiento de carga frontal sea mejor si es simplemente equivalente al tratamiento anterior en todos los aspectos es en las pruebas: en eventos cardíacos y de otro tipo después del inicio del tratamiento trombolítico (15 hipótesis).

Los p$-$values individuales se informan tal como están, sin advertencia alguna sobre su interpretación y los autores concluyen que\\

''\textit{En comparación con el tratamiento con APSAC, a pesar de más reoclusiones tempranas, el curso clínico con el tratamiento con rt-PA es más favorable con menos complicaciones hemorrágicas y una tasa de mortalidad hospitalaria sustancialmente más baja, presumiblemente debido a una mejor permeabilidad temprana de la arteria relacionada con el infarto}''.\\

La afirmación sobre la mortalidad se basa en un valor de p de 0,0095.\\
Considere ahora la cuarta familia, que contiene la comparación de mortalidad y otras 14 comparaciones. Las posiciones ordenadas para las 15 comparaciones realizadas son \begin{align*}
0001, 0,0004, 0,0019, 0,0095, 0,0201, 0,0278, 0,0298,\\ 0,0344, 0.0459, 0.3240, 0.4262, 0.5719, 0.6528, 0.7590, 1.000.
\end{align*}
Controlando el FWER en 0.05, el enfoque de Bonferroni, usando $0.05 / 15 = 0.0033$ rechazamos las tres hipótesis correspondientes a los valores $p$ más pequeños. Y usando el procedimiento de control de FDR considerando el enfoque $BH$ con $q^* = 0.05$ rechazamos las cuatro hipótesis que tienen valores $p$ menores o iguales a $0.013$.

\begin{table}[H]
\begin{tabular}{ccccccccccc}
\hline
\hline
$H_{0i}$ & p$-$values & i &  Umbral BH & Umbral Bonferroni & Rechazo BH& Rechazo Bonferroni\\
\hline
\hline
1 &   0.0001&  1& 0.0034&       0.0033&       TRUE&               TRUE\\
2 &   0.0004&  2& 0.0067&       0.0033&       TRUE&               TRUE\\
3 &   0.0019&  3& 0.0100&       0.0033&       TRUE&               TRUE\\
4 &   0.0095&  4& 0.0134&       0.0033&       TRUE&              FALSE\\
5 &   0.0201&  5& 0.0167&       0.0033&      FALSE&              FALSE\\
6 &   0.0278&  6& 0.0200&       0.0033&      FALSE&              FALSE\\
7 &   0.0298&  7& 0.0234&       0.0033&      FALSE&              FALSE\\
8 &   0.0344&  8& 0.0267&       0.0033&      FALSE&              FALSE\\
9 &   0.0459&  9& 0.0300&       0.0033&      FALSE&              FALSE\\
10&   0.3240& 10& 0.0334&       0.0033&      FALSE&              FALSE\\
11&   0.4262& 11& 0.0367&       0.0033&      FALSE&              FALSE\\
12&   0.5719& 12& 0.0400&       0.0033&      FALSE&              FALSE\\
13&   0.6528& 13& 0.0434&       0.0033&      FALSE&              FALSE\\
14&   0.7590& 14& 0.0467&       0.0033&      FALSE&              FALSE\\
15&   1.0000& 15& 0.0500&       0.0033&      FALSE&              FALSE\\
\hline
\hline
\end{tabular}
\end{table}

Las primeras tres hipótesis corresponden a una reacción alérgica reducida y a dos aspectos diferentes del sangrado; no incluyen la comparación de mortalidad. Por tanto, la afirmación sobre una reducción significativa de la mortalidad no está justificada desde el punto de vista clásico. Pero controlando el FDR rechazamos la hipótesis 4,  ahora con la confianza apropiada las afirmaciones sobre la disminución de la mortalidad, de las que antes no teníamos pruebas suficientemente sólidas.

\subsubsection*{Comparación de la potencia de FDR y FWER}


\subsection*{The q value }
\textit{Optional.}

\subsection*{Aplicación del FDR en la ciencia de datos}
\begin{itemize}
\item FDR en la selección del modelo.
\item FDR en data mining.
\item FDR universo.
\end{itemize}



\end{document}